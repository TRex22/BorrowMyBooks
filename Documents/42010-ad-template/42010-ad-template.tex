
\documentclass[10pt,oneside]{report}

% The preamble contains definitions and settings

%%% Preamble
\usepackage{amsmath}
\usepackage{amssymb}
\usepackage{latexsym}
\usepackage{graphicx}


%%%
%%% Bibliography
%%%
\usepackage[backend=bibtex8, citestyle=numeric, bibstyle=authortitle]{biblatex}
\bibliography{/Users/rh/refs/preamble.bib,
  /Users/rh/refs/ieee-1471.bib, /Users/rh/refs/cs.bib, /Users/rh/refs/arch.bib}

%%% 
%%% Definitions
%%%
\newcommand{\Fillin}[1]{\textcolor{Red}{$<$#1$>$}}
\newcommand{\HRule}{\rule{\linewidth}{0.5mm}}
\newcommand{\Optional}{\textcolor{Gray}{\textsf(optional)}}
\newcommand{\angles}[1]{$\langle$#1$\rangle$}
\newcommand{\must}[1]{\textcolor{NavyBlue}{$\star$ #1}}
\newcommand{\note}[1]{\small\textsc{note: }\textit{#1}}
\newcommand{\should}[1]{\textcolor{Plum}{$\Box$ #1}}
\newcommand{\std}[1]{\textcolor{Maroon}{ISO/IEC/IEEE~42010,~#1}}
\newcommand{\tbd}[1]{\noindent\textcolor{Red}{\textbf{TBD: }{\textsf{#1}}}}
\newcommand{\working}[1]{\noindent\textcolor{CadetBlue}{\textsf{#1}}}

% \renewcommand{\thesection}{\Alph{section}}
% \renewcommand{\thesubsection}{\alph{subsection}}
% \renewcommand{\thesubsubsection}{\roman{subsection}.}

\parindent 0cm
\parskip 0.3cm

% \topmargin 0.2cm
% \oddsidemargin 1cm
% \evensidemargin 0.5cm
% \textwidth 14cm
% \textheight 20cm
% \pagestyle{fancy}

%%%  Prettier Tables
% \usepackage{booktabs}
%%%  Diagrams
% \usepackage[all]{xy}


%%%
%%%  Colors
%%%
\usepackage{color}
\usepackage[usenames,dvipsnames]{xcolor} % see: http://en.wikibooks.org/wiki/LaTeX/Colors


%%% Fonts
%\usepackage[charter]{mathdesign}
%\usepackage{arev}
%\usepackage{ccfonts,eulervm} \usepackage[T1]{fontenc}
%\usepackage{cmbright}
%\usepackage{concrete}
%\usepackage{kmath,kerkis}
%\usepackage{mathpazo}
%\usepackage{mathptmx}
\usepackage{newcent}
%\usepackage{palatino}
%\usepackage{pxfonts}
%%% end Fonts


%%% Index
%%% \makeindex


%%% Make this last package (always load after biblatex):
\usepackage[pdfstartview=FitH,colorlinks=true,citecolor=ForestGreen,linkcolor=NavyBlue]{hyperref}

%%% end Preamble



\begin{document}

%%%%%%%%%%
\begin{titlepage}

\begin{center}
\HRule

\Large\textmd{Architecture description template \\
  for use with ISO/IEC/IEEE 42010:2011}
\vfill

%%%
%%% Title
%%%
{ \huge{Architecture Description of \\ 
    \Fillin{Architecture Name} for \\
    \Fillin{System of Interest}}} \\[0.4cm]

\large\textmd{``Bare bones'' edition}\\
\large\textmd{version: 2.2}
\vfill

%%%
%%% Author
%%%
Template prepared by: \\
\large\textsf{Rich Hilliard \\
  \href{mailto:r.hilliard@computer.org}{r.hilliard@computer.org}}

\vfill
%%%
%%% License
%%%
Distributed under \\
Creative Commons Attribution 3.0 Unported License. \\
For terms of use see:  \\ 
\url{http://creativecommons.org/licenses/by/3.0/}

\HRule
\end{center}
\end{titlepage}
%%%%%%%%%% end titlepage

\pdfinfo{
  /Title (Architecture description template for use with ISO/IEC/IEEE 42010:2011)
  /Author (Rich Hilliard, r.hilliard@computer.org)
  /Keywords (architecture description, ISO/IEC/IEEE 42010)
}

\pagenumbering{roman}
\setcounter{tocdepth}{3}


%%%%%%%%%%
\chapter*{Using the template}
\addcontentsline{toc}{chapter}{Using the template}
%%%%%%%%%%

ISO/IEC/IEEE 42010, \textit{Systems and software engineering ---
  Architecture description}, defines the contents of an architecture
description (AD)~\cite{ISO42010:2011}. 

Figure~\ref{fig:content} depicts that contents in terms of a UML class
diagram.

The AD template in this document defines places for all required
information and offers the user additional guidance on preparing an
AD.

An AD may take many forms, not prescribed by the Standard: it could be
presented as a document, a set of documents, a collection of models, a
model repository, or in some other form -- as long as the required
content is accessible in some manner. In particular, organization and
ordering of required information is not defined by the Standard.
Thus, headings and subheadings in this template are merely suggestive
-- not required.

\begin{figure}[!h]
\pdfimage width 5in {Core-Realm.png}
\caption{Content model of an architecture description}\label{fig:content}
\end{figure}
 
The template uses a few conventions, as follows.

\must{``Musts'' are items which must be present to satisfy the
  Standard. Musts are marked like this.}

\should{``Shoulds'' are items recommended to be present, but not
  required by the Standard. Shoulds are marked like this.}

Optional items are marked with this: \Optional.

\Fillin{Items} like \Fillin{this} signal names to be filled-in by a
user of the template and used throughout the resulting AD.




%%%%%%%%%%
\section*{License}
\addcontentsline{toc}{section}{License}
%%%%%%%%%%

The \textit{Architecture Description Template} is copyright
\copyright\ 2012--2014 by
\href{http://www.iso-architecture.org/42010/templates/}%
{Rich Hilliard}.

The latest version is always available at: \\
\url{http://www.iso-architecture.org/42010/templates/}.

The template is licensed under a Creative Commons Attribution 3.0
Unported License. For terms of use, see:\\
\url{http://creativecommons.org/licenses/by/3.0/}

\vfill \pdfimage {88x31.png}

This license gives you the user the right to share and remix this work
to create architecture descriptions.  It does not require you to share
the results of your usage, but if your use is non-proprietary, we
encourage you to share your work with others via the WG42 website \\
\url{http://www.iso-architecture.org/42010/}.



%%%%%%%%%%
\section*{Version history}
\addcontentsline{toc}{section}{Version History}
%%%%%%%%%%

This template is based on one originally designed for use with
IEEE~std~1471:2000~\cite{IEEE1471:2000} and now updated for
ISO/IEC/IEEE 42010:2011.  The present document is an enhanced version
of the earlier template, with additional guidance, clarifications and
examples for readers.

\begin{description}
\item[rev 2.2] 7 October 2014, Moved bibliography from \texttt{bibtex}
  to \texttt{biblatex}. Released revision with minor formatting fixes.
\item[rev 2.1a] June 2012, initial release on 42010 website.
\end{description}


%%%%%%%%%%
\subsection*{Editions} 
\addcontentsline{toc}{subsection}{Template editions}
%%%%%%%%%%
This is the ``bare bones'' edition -- it contains exactly only
information items required by the Standard. Other editions meet the
requirements of the Standard and contain additional information used
in various documentation approaches (such as
\cite{Rozanski-Woods:2011,DSA:2010}).


%%%%%%%%%%
\section*{Comments or questions}
\addcontentsline{toc}{section}{Comments}
%%%%%%%%%%
 
Contact the author \href{mailto:r.hilliard@computer.org}%
{Rich Hilliard $\langle$r.hilliard@computer.org$\rangle$} with
comments, suggestions, improvements or questions.

For more information on ISO/IEC/IEEE 42010, visit the website: \\
\url{http://www.iso-architecture.org/42010/}.

\textcolor{Orange}{\Large The template begins here . . .}

%%%
%%% Change page number for main body of text
%%%
\pagenumbering{arabic}


%%%%%%%%%%
\chapter{Introduction}\label{ad:info}
%%%%%%%%%%

This chapter describes introductory information items of the AD,
including identifying and supplementary information.


%%%%%%%%%%
\section{Identifying information}\label{ad:idinfo}
%%%%%%%%%%

\must{Identify the architecture being expressed, such as via an
  \Fillin{Architecture Name}, or as appropriate.}

\must{Identify the \Fillin{System of Interest} for which this is an
  architecture description.}

Following ISO/IEC/IEEE 42010, \textit{system} (or
\textit{system-of-interest}) is a shorthand for any number of things
including man-made systems, software products and services, and
software-intensive systems including ``individual applications,
systems in the traditional sense, subsystems, systems of systems,
product lines, product families, whole enterprises, and other
aggregations of interest''. \std{4.2}

There is also a place for these two information items on the title
page.


%%%%%%%%%%
\section{Supplementary information}\label{ad:supinfo}
%%%%%%%%%%

\must{Provide supplementary information as determined by the project and/or
organization.}

The details of identifying and supplementary information items are not
defined by the Standard. Most organizations or projects will have
their own requirements.

Examples of identifying and supplementary information include: date of
issue and status; authors, reviewers, approving authority, issuing
organization; change history; summary; scope; context; glossary;
version control information; configuration management information and
references. \std{5.2}


%%%%%%%%%%
\section{Other information}
%%%%%%%%%%

Although views and models are the primary form of organization in an
architecture description, an AD may also contain information not part
of any architecture view or model. 

Examples of types of architecture information that might not be part
of any architecture view are:
\begin{itemize}
\item system or architecture overview
\item reader's guide to this AD
\item results of architecture evaluations
\item rationale for key decisions
\item view and model correspondences 
\end{itemize}

This section discusses these forms of information: overviews,
architecture evaluations and rationale.



\subsection{Overview \Optional}
Provide an overview of the architecture being described,
its essential points, and a summary of the
\Fillin{System of Interest}.

An Overview could include sections for Purpose, Scope and Context of the architecture.

Provide an overview of the remainder of this AD as a guide for its
readers.

Recognizing the key role of stakeholders and concerns
(see~\S\ref{sec:snc}) for the contents of the AD, consider Overview(s)
organized by stakeholders and/or concerns.


%%%%%%%%%%
\subsection{Architecture evaluations}
%%%%%%%%%%

\must{Include results from any evaluations of the
  \Fillin{Architecture Name} being documented.}


%%%%%%%%%%
\subsection{Rationale for key decisions}
%%%%%%%%%%

\must{An architecture description shall include rationale for each
  decision considered to be a key architecture decision (per
  \std{5.8.2}).}

See~\S\ref{ch:decisions} for further guidance about decisions and rationale.


%%%%%%%%%%
\chapter{Stakeholders and concerns}\label{sec:snc}
%%%%%%%%%%

This chapter contains information items for stakeholders of the 
architecture, the stakeholders' concerns for that architecture, and
the traceability of concerns to stakeholders. See also:~\std{5.3}

%%%%%%%%%%
\section{Stakeholders}\label{ad:stakeholders}
%%%%%%%%%%

\must{Identify and describe the stakeholders for the architecture.}

  Stakeholders may be individuals (e.g., ``Joe the CEO''), groups
  (e.g., ``power users of our product''), organizations (e.g., ``the
  Quality Assurance Department'' or ``the NSA'') as well as classes of
  same.

  The stakeholders have areas of interest, called \textit{concerns},
  that are considered fundamental to the architecture of the
  system-of-interest. See~\S\ref{ad:concerns}.

\must{The following stakeholders must be considered when preparing an
  AD.  When applicable to the \Fillin{System of Interest}, they must
  be identified in the architecture description: users, operators,
  acquirers, owners, suppliers, developers, builders and maintainers.}

  These names are not required to be used; stakeholder names should be
  chosen as appropriate to the \Fillin{System of Interest}, the
  project and/or organization.


%%%%%%%%%%
\section{Concerns}\label{ad:concerns}
%%%%%%%%%%

\must{Identify the concerns considered fundamental to the architecture
  of \Fillin{System of Interest}.}

\must{Consider the following concerns, and include them in the AD when
  applicable:}

\begin{itemize}
\item What are the purpose(s) of the system-of-interest?
\item What is the suitability of the architecture for achieving the
  system-of-interest's purpose(s)?
\item How feasible is it to construct and deploy the
  system-of-interest?
\item What are the potential risks and impacts of the
  system-of-interest to its stakeholders throughout its life cycle?
\item How is the system-of-interest to be maintained and evolved?
\end{itemize}

These concern statements are not required to be used; concern
statements should be chosen as appropriate to the
\Fillin{System of Interest}, the project and/or organization.

For further guidance on concerns, see~\S\ref{vp:concerns}.


%%%%%%%%%%
\section{Concern--Stakeholder Traceability}
%%%%%%%%%%

\must{Associate each identified concern from~\S\ref{ad:concerns} with
  the identified stakeholders from~\S\ref{ad:stakeholders} having that
  concern.}

This association can be recorded via a simple table or other
depiction.

\begin{table}[h]
\caption{Example showing association of stakeholders to
  concerns in an AD}\label{table:SxC}
\begin{tabular}{ l | c | c | c | c }
   & \textsf{Stakeholder~1} & \textsf{Stakeholder~2} & \textsf{Stakeholder~3} &. . . \\
\hline
  \textsf{Concern~1} & -- & \textsf{x} & \textsf{x} & . . . \\
  \textsf{Concern~2} & \textsf{x} & -- & \textsf{x} & . . . \\
  \textsf{Concern~3} & -- & \textsf{x} & \textsf{x} & . . . \\
  . . . & & &  \\
\hline
\end{tabular}
\end{table}

%%%%%%%%%%
\chapter{Viewpoints+}\label{ad:vps}
%%%%%%%%%%

An AD contains multiple architecture views; each view adheres to the
conventions of an \textit{architecture viewpoint}. This chapter
describes the requirements on documenting viewpoints for an AD.

\must{Include a specification for each architecture viewpoint used in
  this AD.}

\must{Viewpoints must be chosen for the AD such that each identified
  concern from~\S\ref{ad:concerns} is framed by at least one
  viewpoint.}

\must{Provide a rationale for each viewpoint used.} 

Rationale could include discussion in terms of its stakeholders, the
concerns framed by the viewpoint, relevance of its model kinds and
modeling conventions.

\must{Each architecture viewpoint used in the AD must be specified in
  accordance with the provisions of \std{7}.}

A detailed template for specifying viewpoints in accordance with the
Standard is included in ~\S\ref{vp:template} below.

\note{The latest version of the viewpoint template can be found at  \\
  \url{http://www.iso-architecture.org/42010/templates/}.}

Repeat and fill-in viewpoint template as needed for each viewpoint
used in the AD.


An AD contains one or more architecture views and an architecture
viewpoint definition for each view. There is no required ordering of
the views or viewpoints within an AD.  Readers of the AD will need to
refer to the viewpoint specifications to understand the subject of a
view, its notations, models and the modeling conventions used.  Given
a set of views ($v_i$) and their viewpoints ($\mathsf{VP}_i$), the
architect might consider the following possible arrangements:

\begin{itemize}
\item Viewpoints, first: $\mathsf{VP}_i$, followed by the views: $v_i$
\item Interleaved views with their viewpoints: $v_i$, $\mathsf{VP}_i$,
  $v_j$, $\mathsf{VP}_j$, \ldots
\item Views up front: $v_i$ with the viewpoints deferred to
  appendices, $\mathsf{VP}_i$
\end{itemize}


%%% include Viewpoint Template here:
%%% Version 2.1b %%%

%%%%%%%%%%
\section{\Fillin{Viewpoint Name}}\label{vp:template}
%%%%%%%%%%

\must{Provide the name for the viewpoint.}

If there are any synonyms or other common names by which this viewpoint is
known or used, record them here.


%%%%%%%%%%
\section{Overview} 
%%%%%%%%%%

Provide an abstract or brief overview of the viewpoint. 

Describe the viewpoint's key features.


%%%%%%%%%%
\section{Concerns and stakeholders} 
%%%%%%%%%%

Architects looking for an architecture viewpoint suitable for their
purposes often use the identified concerns and typical stakeholders to
guide them in their search.  Therefore it is important (and required
by the Standard) to document the concerns and stakeholders for which a
viewpoint is intended.

%%%%%%%%%%
\subsection{Concerns}\label{vp:concerns}
%%%%%%%%%%

\must{Provide a listing of architecture-relevant concerns to be framed by
this architecture viewpoint per \std{7a}.}

Describe each concern.

Concerns name ``areas of interest'' in a system.

\note{Following ISO/IEC/IEEE 42010, \textbf{system} is a shorthand for
  any number of things including man-made systems, software products
  and services, and software-intensive systems such as ``individual
  applications, systems in the traditional sense, subsystems, systems
  of systems, product lines, product families, whole enterprises, and
  other aggregations of interest''.}

Concerns may be very general (e.g., \textit{Reliability}) or quite
specific (\textit{e.g., How does the system handle network latency?}).
  
Concerns identified in this section are critical information for an
architect because they help her decide when this viewpoint will be
useful.

When used in an architecture description, the viewpoint becomes a
``contract'' between the architect and stakeholders that these
concerns will be addressed in the view resulting from this viewpoint.

It can be helpful to express concerns \emph{in the form of questions}
that views resulting from that viewpoint will be able to answer. E.g.,
\begin{itemize}
\item \textit{How does the system manage faults?}
\item \textit{What services does the system provide?}
\end{itemize}

\note{``In the form of a question'' is inspired by the television quiz
  show, \textit{Jeopardy!}}
 
\std{5.3} contains a candidate list of concerns that must be considered
when producing an architecture description. These can be considered
here for their relevance to the viewpoint being specified:
\begin{itemize}
\item What are the purpose(s) of the system-of-interest?
\item What is the suitability of the architecture for achieving the
  system-of-interest's purpose(s)?
\item How feasible is it to construct and deploy the
  system-of-interest?
\item What are the potential risks and impacts of the
  system-of-interest to its stakeholders throughout its life cycle?
\item How is the system-of-interest to be maintained and evolved?
\end{itemize}

See also: \std{4.2.3}.

%%%%%%%%%%
\subsection{Typical stakeholders} 
%%%%%%%%%%

\must{Provide a listing of the typical stakeholders of a system who
  are in the potential audience for views of this kind, per \std{7b}.}

Typical stakeholders would include those likely to read such views
and/or those who need to use the results of this view for another
task.

Stakeholders to consider include:
\begin{itemize}
\item users of a system; 
\item operators of a system; 
\item acquirers of a system;
\item owners of a system; 
\item suppliers of a system; 
\item developers of a system; 
\item builders of a system; 
\item maintainers of a system.
\end{itemize}

%%%%%%%%%%
\subsection{``Anti-concerns'' \Optional} 
%%%%%%%%%%

It may be helpful to architects and stakeholders to
document the kinds of issues for which this viewpoint is \emph{not
  appropriate or not particularly useful}.

Identifying the ``anti-concerns'' of a given notation or approach may
be a good antidote for certain overly used models and notations.

% \tbd{Examples!}



%%%%%%%%%%
\section{Model kinds+}\label{mk:list}
%%%%%%%%%%

\must{Identify each model kind used in the viewpoint per \std{7c}.}

In the Standard, each architecture view consists of multiple
architecture models. Each model is governed by a \textit{model kind}
which establishes the notations, conventions and rules for models of
that type.  See: \std{4.2.5, 5.5 and 5.6}.

Repeat the next section for each model kind listed here the viewpoint
being specified.


%%%%%%%%%%
\section{\Fillin{Model Kind Name}}\label{vp:mk}
%%%%%%%%%%

\must{Identify the model kind.}


%%%%%%%%%%
\subsection{\Fillin{Model Kind Name} conventions} 
%%%%%%%%%%

\must{Describe the conventions for models of this kind.}

Conventions include languages, notations, modeling techniques,
analytical methods and other operations. These are key modeling
resources that the model kind makes available to architects and
determine the vocabularies for constructing models of the kind and
therefore, how those models are interpreted and used.

It can be useful to separate these conventions into a \emph{language
  part}: in terms of a metamodel or specification of notation to be
used and a \emph{process part}: to describe modeling techniques used
to create the models and methods which can be used on the models that
result.  These include operations on models of the model kind.

The remainder of this section focuses on the language part. The next
section focuses on the process part.

The Standard does not prescribe \emph{how} modeling conventions are to
be documented.  The conventions could be defined:
\begin{description}
\item[I)] by reference to an existing notation or language (such as
  SADT, UML or an architecture description language such as ArchiMate
  or SysML) or to an existing technique (such as $M/M/4$ queues);
\item[II)] by presenting a metamodel defining its core constructs;
\item[III)] via a template for users to fill in;
\item[IV)] by some combination of these methods or in some other
  manner.
\end{description}

Further guidance on methods I) through III) is provided below.
 
Sometimes conventions are applicable across more than one model kind
-- it is not necessary to provide a separate set of conventions, a
metamodel, notations, or operations for each, when a single
specification is adequate.


%%%%%%%%%%
\subsubsection{I) Model kind languages or notations \Optional}
%%%%%%%%%%

Identify or define the notation used in models of the kind.

Identify an existing notation or model language or define one that can
be used for models of this model kind. Describe its syntax, semantics,
tool support, as needed.


%%%%%%%%%%
\subsubsection{II) Model kind metamodel \Optional} 
%%%%%%%%%%

A metamodel presents the AD elements that constitute the
vocabulary of a model kind, and their rules of combination. There are
different ways of representing metamodels (such as UML class diagrams, OWL,
eCore). The metamodel should present:
\begin{description}
\item[entities] What are the major sorts of conceptual elements that
  are present in models of this kind?
\item[attributes] What properties do entities possess in models of
  this kind?
\item[relationships] What relations are defined among entities in
  models of this kind?
\item[constraints] What constraints are there on entities, attributes
  and/or relationships and their combinations in models of this kind?
\end{description}

\note{Metamodel constraints should not be confused with architecture
  constraints that apply to the subject being modeled, not the
  notations used.}

In the terms of the Standard, entities, attributes, relationships are
\textit{AD elements} per \std{3.4, 4.2.5 and 5.7}.

In the \textit{Views-and-Beyond} approach~\cite{DSA:2010}, each
viewtype (which is similar to a viewpoint) is specified by a set of
elements, properties, and relations (which correspond to entities,
attributes and relationships here, respectively).

When a viewpoint specifies multiple model kinds it can be useful to
specify a single viewpoint metamodel unifying the definition of the
model kinds and the expression of correspondence rules.  When defining
an architecture framework, it may be helpful to use a single metamodel
to express multiple, related viewpoints and model kinds.

% \tbd{EXAMPLE -- In \cite{Hilliard:1999} and earlier work, we said that
%   all views are built from primitives called components, connections
%   and constraints which basically gives views a graph structure with
%   components as nodes and two types of edges (connections and
%   constraints). There are two issues with this: (\textit{1})
%   components and \textit{connectors} have taken on a specialized
%   meaning from the work by CMU and others \cite{Shaw-Garlan:1996};
%   (\textit{2}) this ur-ontology may be over-commiting for some views.}


%%%%%%%%%%
\subsubsection{III) Model kind templates \Optional}
%%%%%%%%%%

Provide a template or form specifying the format and/or content of
models of this model kind.

%% \tbd{EXAMPLE} 


%%%%%%%%%%
\subsection{\Fillin{Model Kind Name} operations \Optional} 
%%%%%%%%%%

Specify operations defined on models of this kind.

See~\S\ref{Opns} for further guidance.


%%%%%%%%%%
\subsection{\Fillin{Model Kind Name} correspondence rules}
%%%%%%%%%%

\must{Document any correspondence rules associated with the model
  kind.}

See~\S\ref{CRs} for further guidance.


%%%%%%%%%%
\section{Operations on views}\label{Opns}
%%%%%%%%%%

Operations define the methods to be applied to views and their models.
Types of operations include:

\begin{description}

\item[construction methods] are the means by which views are
  constructed under this viewpoint. These operations could be in the
  form of process guidance (how to start, what to do next); or work
  product guidance (templates for views of this type). Construction
  techniques may also be heuristic: identifying styles, patterns, or
  other idioms to apply in the synthesis of the view.

\item[interpretation methods] which guide readers to understanding
  and interpreting architecture views and their models.

\item[analysis methods] are used to check, reason about, transform,
  predict, and evaluate architectural results from this view,
  including operations which refer to model correspondence rules.

\item[implementation methods] are the means by which to design and
  build systems using this view.

\end{description}

Another approach to categorizing operations is from Finkelstein et
al. \cite{Finkelstein+1992}. The \emph{work plan} for a viewpoint
defines 4 kinds of actions (on the view representations):
\textit{assembly actions} which contains the actions available to the
developer to build a specification; \textit{check actions} which
contains the actions available to the developer to check the
consistency of the specification; \textit{viewpoint actions} which
create new viewpoints as development proceeds; \textit{guide actions}
which provide the developer with guidance on what to do and when.


%%%%%%%%%%
\section{Correspondence rules}\label{CRs}
%%%%%%%%%%

\must{Document any correspondence rules defined by this viewpoint or
  its model kinds.}

Usually, these rules will be across models or across views since,
constraints within a model kind will have been specified as part of
the conventions of that model kind.

See: \std{4.2.6 and 5.7}

%%\tbd{examples or specs}

%%%%%%%%%%
\section{Examples \Optional} 
%%%%%%%%%%

Provide helpful examples of use of the viewpoint for the reader
(architects and other stakeholders).


%%%%%%%%%%
\section{Notes \Optional} 
%%%%%%%%%%

Provide any additional information that users of the viewpoint may
need or find helpful.


%%%%%%%%%%
\section{Sources} 
%%%%%%%%%%

\must{Identify sources for this architecture viewpoint, if any,
  including author, history, bibliographic references, prior art, per
  \std{7e}.}




%%%%%%%%%%
\chapter{Views+}
%%%%%%%%%%

Much of the material in an AD is presented through its architecture views.
Each view follows the conventions of its governing viewpoint.
A view is made up of architecture models.

\must{Include an architecture view for each viewpoint selected
  in~\S\ref{ad:vps}.}

Repeat and complete the following section for each architecture view
in the AD.


%%%%%%%%%%
\section{View: \Fillin{View Name}}
%%%%%%%%%%

\must{Give the architecture view a \Fillin{View Name}.}

\must{Provide any identifying and supplementary information about
  \Fillin{View Name}.}

The details of this information will be as specified by the
organization and/or project. See~\S\ref{ad:info} for examples of
identifying and supplementary information.

Views have their own identifying and supplementary information
distinct from ADs because they may be developed and evolve separately
over the lifetime of a project.

\must{Identify the viewpoint governing this view from among those
  identified in~\S\ref{ad:vps}.}

See also: \std{5.5}


%%%%%%%%%%
\subsection{Models+}
%%%%%%%%%%

An architecture view is composed of one or more architecture models.

\must{Provide one or more architecture models adhering to the
  governing viewpoint.}

\must{The models must address all of the concerns framed by the view's
  governing viewpoint and cover the whole system from that viewpoint.}

Repeat the section below for each model.

\subsection{\Fillin{Model Name}}

\must{Each architecture model shall include version identification as
  specified by the organization and/or project.}

\must{Each architecture model shall identify its governing model kind
  and adhere to the conventions of that model kind
  from~\S\ref{vp:mk}.}

See \std{5.4}.

An architecture model may be a part of more than one architecture
view. This enables sharing of details and addressing distinct but
related concerns without redundancy. Other uses of multiple models:
aspect-oriented style of architecture description: architecture models
shared across architecture views can be used to express architectural
perspectives~\cite{Rozanski-Woods:2011} and architecture
textures~\cite{Ran:2000}. Architecture models can be used as
containers for applying architecture patterns or architecture styles
to express fundamental schemes (such as layers, three-tier,
peer-to-peer, model-view-controller) within architecture views.


%%%%%%%%%%
\subsection{Known Issues with View}\label{known}
%%%%%%%%%%

\must{Document any discrepancies between the view and its viewpoint
  conventions. Each architecture view must adhere to the conventions
  of its governing architecture viewpoint.}

Known issues could include: inconsistencies, items to be completed,
open or unresolved issues, exceptions and deviations from the
conventions established by the viewpoint.  Open issues can lead to
decisions to be made. Exceptions and deviations can be documented as
decision outcomes and rationale.



%%%%%%%%%%
\chapter{Consistency and correspondences}
%%%%%%%%%%

This chapter describes consistency requirements, recording of known
inconsistencies in an AD, and the use and documentation of
correspondences and correspondence rules.


\section{Known inconsistencies}
%%%%%%%%%%

\must{Record any known inconsistencies in the AD.}

Although consistent ADs obviously are to be preferred,
it is sometimes infeasible or impractical to resolve all
inconsistencies for reasons of time, effort, or insufficient
information.

\should{An architecture description should include an analysis of
  consistency of its architecture models and its views.}


\section{Correspondences in the AD}
%%%%%%%%%%

\must{Identify each correspondence in the AD and its participating AD
  elements. Identify any correspondence rules governing }

Correspondences are used to express, record, enforce and analyze
consistency between models, views and other AD elements within an
architecture description, between ADs, or between an AD and other
forms of documentation.

AD elements include instances of stakeholders, concerns, viewpoints and
views, model kinds and models, decisions and rationales. Constructs
introduced by viewpoints and model kinds are also AD elements.

Correspondences are n-ary mathematical relations. Correspondences can
be depicted via tables, via links, or via other forms of association
(such as in UML).



\section{Correspondence rules}
%%%%%%%%%%

\must{Identify each correspondence rule applying to the AD.}

Correspondence rules can be introduced by the AD, by one of its
viewpoints, or from an architecture framework or architecture
description language being used.


\must{For each identified correspondence rule, record whether the rule
  holds (is satisfied) or otherwise record all known violations.}


%%%%%%%%%%
\appendix
\chapter{Architecture decisions and rationale}\label{ch:decisions}
%%%%%%%%%%

It is not required by the Standard to capture architecture
decisions. This section describes recommendations (``shoulds'') for
their recording.

%%%%%%%%%%
\section{Decisions}
%%%%%%%%%%

\should{Provide evidence of consideration of alternatives and the
  rationale for the choices made.}

\should{Record architecture decisions considered to be key to the
  architecture of \Fillin{System of Interest}.}

Areas to consider to selecting key decisions include those:
\begin{itemize}
\item affecting key stakeholders or many stakeholders
\item essential to project planning and management
\item expensive to enforce or implement
\item highly sensitive to changes or costly to change
\item involving intricate or non-obvious reasoning
\item pertaining to architecturally significant requirements
\item requiring major expenditures of time or effort to make
\item resulting in capital expenditures or indirect costs
\end{itemize}

\should{When recording decisions, the following information items should be considered:}
\begin{itemize}
\item unique identifier for the decision 
\item statement of the decision
\item correspondences or linkages concerns to which it pertains
\item owner of the decision
\item correspondences or linkages to affected AD elements
\item rationale linked to the decision
\item forces and constraints on the decision
\item assumptions influencing the decision
\item considered alternatives and their potential consequences
\end{itemize}

See \cite{DFAD:2011} and references there for various
approaches to documenting decisions compatible with the Standard.


%%%%%%%%%%
\center\textcolor{Orange}{\Large The template ends here!}
%%%%%%%%%%


%%%%%%%%%% Bibliography
\printbibliography
%%%%%%%%%%

%%%%%%%%%%
% \include{index}
% \addcontentsline{toc}{chapter}{Index}
%%%%%%%%%%


\newpage
%%%%%%%%%%
\tableofcontents
% \listoffigures
% \listoftables
%%%%%%%%%%

\end{document}
